
\section{The Design of Data Exchange} 

For various reasons, many x-ray techniques developed at synchrotron facilities
around the world are unable to directly compare results due to their 
inability to exchange data and software tools. The aim of Data Exchange is to
define a simple file format offering few basic rules and allowing each
community to extend and add technique specific components. The goal is to
provide extensibility in defining data, meta data and
provenance information in a simple way that can be easily adopted by various
x-ray techniques.

The Data Exchange format is implemented using Hierarchical Data Format 5
(HDF5), which offers platform-independent binary data storage with optional
compression, hierarchical data ordering, and self-describing tags so that one
can examine a HDF5 file's contents with no knowledge of how the file writing
program was coded.

The aim and the scope of Data Exchange is very similar to the Coherent X-ray
Imaging  Data Bank file format (CXI), so whenever possible we will use the
same conventions, name tags, and reference system. This document is using a
similar diagram definition set by Filipe R. N. C. Maia in "CXI file format"
(\url{http://cxidb.org/cxi.html}), with a few minor modifications for Data
Exchange definitions.

The core principle of Data Exchange is that it must be simple enough that it
is not necessary to use a support library beyond core HDF5. The simplicity of
Data Exchange makes it easy for anyone to either look at an example file 
using h5dump or HDFView, or to look at example code in language X, and 
then create their own read and write routines in language Y.

The simplest Data Exchange file provides information
sufficient to share a multidimensional data array. In
this simplest form, Data Exchange implements only one "exchange"
group. The "exchange" definition is designed to allow for simple exchange of
images, spectra, and other forms of beamline detector data with a minimum of
fields. This definition is essentially a technique-agnostic format for
exchanging data with others. Data Exchange is also designed to be extended
to include technique-specific data and metadata.  This is
achieved by providing optional, but clearly defined, metadata components to
the base definition.

\subsection{HDF5}

The HDF5 format is the basis of the Data Exchange format. Data Exchange, like
CXI, is not a completely new file format, but simply a set of rules designed
to create HDF5 files with a common structure to allow a uniform and
consistent interpretation of such files.

HDF5 was chosen as the basis because it is a widely used high performance
scientific data format which many programs can already, at least partially,
read and write. It also brings with it the almost automatic fulfillment of the
Data Exchange requirements, i.e. simplicity, flexibility and extensibility.
HDF5 version 1.8 or higher is required as previous versions don't support all
features required by Data Exchange.

\subsection{Data types}

Data Exchange uses the same CXI convention for data types as defined at 
\url{http://cxidb.org/cxi.html} using HDF5 native datatypes.  The data should
be saved in the same format as it was created/acquired. For example CCD images
acquired as 16 bit integers should be saved using the {\tt H5T\_NATIVE\_SHORT}
HDF5 type. In this way all cross platform big-little endian issues reading and
writing files are eliminated.

\subsection{Root Level Structure}
\label{DataExchange}

While HDF5 gives great flexibility in data storage, straightforward file
readability and exchange requires adhering to an agreed-upon naming and
organizational convention. To achieve this goal, Data Exchange adopts a
layered approach by defining a set of \emph{mandatory} and
{\tt optional} fields. 

The general structure of a Data Exchange file is shown in table~\ref{tab:genrules}.
The most basic file must have an "implements" string, and an "exchange" group at
the root level/group of the HDF5 file. Optional "measurement" and "provenance"
groups are also defined. Beyond this, additional groups may be added to meet
individual needs, with guidelines suggesting the best structure.

\begin{table}[h!]\sffamily
\centering
\footnotesize
\caption{Data Exchange Top Level Members}
\label{tab:genrules}
\rowcolors{1}{white}{tableBlue}
\begin{tabular}{l l l}
\toprule
\bfseries Member     & \bfseries Type & \bfseries Example \\
\midrule

\emph{implements} & string dataset & "exchange:measurement:provenance" \\
\hyperref[sec:exchange]{\emph{exchange}} &  group  & \\
\hyperref[table:measurement]{\tt{measurement}} &  group  & \\
\hyperref[table:provenance]{\tt{provenance}} & group & \\
\bottomrule
\end{tabular}
\end{table}

\begin{description}
\item[\emph{implements}] \hfill \\
{Mandatory scalar string dataset in the root of the HDF5 file whose
value is a colon separated list that shows which components are present in the
file. All components listed in the \emph{implements} string are to be groups
placed in the HDF5 file at the root level/group. In a minimal Data
Exchange file, the only mandatory item in this list is \emph{exchange}. A more
general Data Exchange file also contain {\tt{measurement}} and possibly  
{\tt{provenance}}, in which case the implements string would be:
"\emph{exchange}: {\tt{measurement}}: {\tt{provenance}}"}

\item[\emph{exchange}] \hfill \\
{Mandatory group containing one or more arrays that
represent the most basic version of the data, such as raw or normalized
optical density maps or a elemental signal map.  \emph{Exchange\_$N$} is
used when more than one core dataset or derived datasets are saved in the file.
The \emph{exchange} implementation for specific techniques are defined
in separate sections in the Reference Guide.}

\item[\tt measurement] \hfill \\
{Optional group containing the measurement made on the
sample.  {\tt Measurement} contains information about the {\tt sample}  and
the  {\tt instrument}. {\tt Measurement\_$N$} is used when more than one
measurement is stored in the same file.}

\item[\tt provenance] \hfill \\
{Optional group containing information about the status
of each processing step.}

\end{description}

In a  Data Exchange file, each dataset has a unit defined using the
{\tt units} attribute. {\tt units}  is not mandatory - if omitted, the default
unit as defined in Appendix~\ref{appendix:units} is used.

The detailed rules about how to store datasets within the exchange group
are best shown through examples in the next section. Detailed reference information
can be found in the \hyperref[sec:corereference]{Data Exchange Core Reference} section.

