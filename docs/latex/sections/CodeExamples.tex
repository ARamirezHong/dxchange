\section{Code Examples}

% \listoffigures

Below is the Python code to generate the Data Exchange files decribed in the diagrams in Fig.  \ref{fig:Minimal1}, \ref{fig:MinimalTomo0}, \ref{fig:MinimalTomo1} (right), \ref{fig:MinimalTomo2} (right), \ref{fig:MinimalTomo3} (left) as well as full implementation for tomography.


All the code examples as well as the resulting Data Exchange files are available from \url{http://www.aps.anl.gov/DataExchange/}.

\hypersetup{linkcolor = white}
 
\newpage
\subsection{Creating Data Exchange files using Pyhton}

\lstset{caption={Python Code for Fig. \ref{fig:Minimal1}}}
\lstinputlisting{code/python/DataExchange-example0.py}


\newpage
\lstset{caption={Python Code for Fig. \ref{fig:MinimalTomo0}}}
\lstinputlisting{code/python/DataExchange-example1.py}

\newpage
\lstset{caption={Python Code for Fig. \ref{fig:MinimalTomo1} (right)}}
\lstinputlisting{code/python/DataExchange-example2.py}

\newpage
\lstset{caption={Python Code for Fig. \ref{fig:MinimalTomo2} (right)}}
\lstinputlisting{code/python/DataExchange-example3.py}

\newpage
\lstset{caption={Python Code for Fig. \ref{fig:MinimalTomo3} (right)}}
\lstinputlisting{code/python/DataExchange-example4.py}

\newpage
\lstset{caption={Python Code for Data Exchange full implementation for tomography}}
\lstinputlisting{code/python/DataExchange-example5.py}

\newpage
\lstset{caption={Python Code for Data Exchange full implementation for tomography - includes provenance}}
\lstinputlisting{code/python/DataExchange-example6.py}

\hypersetup{linkcolor = softBlue}

\subsection{Creating Data Exchange files using C++}
