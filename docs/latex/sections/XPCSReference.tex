%****************************************************************************
\section{Data Exchange for X-ray Photon Correlation}
\label{exchange:photon}

In x-ray photon correlation the arrays representing the most basic version of the data include .........

\begin{table}[h!]\sffamily \footnotesize
\caption{exchange class members for x-ray photon correlation}

\rowcolors{1}{white}{tableBlue}
\begin{tabular}{p{2.0cm} p{3.0cm}  p{6.5cm} }
\toprule
\bfseries Member     & \bfseries Type & \bfseries Example \\
\midrule
title  & string & "raw photon correlation data" \\
\emph{data}  &  array & see \ref{table:attribphoton} for attributes \\
\bottomrule
\end{tabular}
\end{table}

\member{title}{This is the data title.}

\member{\emph{data}}{An x-ray photon correlation data set consists of ....... }

%****************************************************************************
\subsection{Instrument specific for x-ray photon correlation}
\label{instrument:photon}

\subsubsection{Detector}
\label{table:detectorphoton}

This class holds information about the detector used during the experiment. If  more than one detector are used they will be all listed as detector\_$N$.


\begin{table}[h!]\sffamily \footnotesize
\caption{X-ray photon correlation detector class members}

\rowcolors{1}{white}{tableBlue}
\begin{tabular}{p{3.5cm} p{4.7cm}  p{4.5cm} }
\toprule
\bfseries Member     & \bfseries Type & \bfseries Example \\
\midrule
manufacturer & string & "CooKe Corporation" \\   
model & string &  "pco dimax" \\
serial\_number & string &  "1234XW2" \\  
bit\_depth & integer & 12 \\     
x\_pixel\_size & float & 6.7e-6 \\
y\_pixel\_size & float & 6.7e-6 \\
x\_dimension & integer & 2048 \\
y\_dimension & integer & 2048 \\
x\_binning & integer & 1 \\
y\_binning & integer & 1 \\
operating\_temperature & float &  270 \\     
exposure\_time & float & 1.7e-3 \\
exposure\_period & float & 10.0 \\   
frame\_rate & integer &  2 \\
distance     & float & 5.7e-3\\
data\_flat & 3D array & variable (see Tab. \ref{table:attrib} for attrib.)  \\
roi & \hyperref[table:roi]{roi class} & \\
efficiency & float & 99.95 \\
adu\_per\_photon & float & 5.0 \\
gain & float & 1.0 \\
basis\_vectors & float array & length \\ 
corner\_position & 3 floats & length \\
blemish\_mask & 2D array of integers & variable (see Tab. \ref{table:attrib} for attrib.)  \\
kinetics & \hyperref[table:kinetics]{kinetics class} & \\
geometry & string & "TRANSMISSION" \\
\bottomrule
\end{tabular}
\end{table}

\member{manufacturer}{The detector manufacturer.}

\member{model}{The detector model.}

\member{serial\_number}{The detector serial number .}
     
\member{bit\_depth}{The detector bit depth.}

\member{x\_pixel\_size, y\_pixel\_size}{Physical detector pixel size (m).}

\member{x\_dimension, y\_dimension}{The detector horiz./vertical dimension.}

\member{x\_binning, y\_binning}{If the data are collected binning the detector x\_binning and y\_binning store the binning factor.}

\member{operating\_temperature}{The detector operating temperature (K).}

\member{exposure\_time}{The detector exposure time (s).}

\member{exposure\_period}{Time from the beginning of an exposure to the beginning of the next exposure (s).}

\member{frame\_rate}{The detector frame rate (fps). This parameter is set for fly scan}

\member{distance}{The detector distance from the sample.}

\member{data\_flat}{The dark field and white fields must have the same dimensions as the collected images.  Data\_flat attributes, if used, are defined in Table \ref{table:attrib}.}

\member{roi}{The detector selected Region Of Interest (ROI).}

\member{efficiency}{The efficiency of the detector.}

\member{adu\_per\_photon}{The ADU per photon.}

\member{gain}{Detector gain setting.}

\member{basis\_vectors}{A matrix with the basis vectors of the
  detector data. For more details see \ref{ccd_orientation}.
}

\member{corner\_position}{The x, y and z coordinates of the corner of the first data element. For more details see \ref{ccd_orientation}.
}

\member{blemish\_mask}{Blemish mask labeling dead pixels in the detector. It's a 2D array of the same dimensions as the full detector, with 0 labeling bad pixels and 1 labeling good pixels.}

\member{kinetics}{Kinetics detector properties.}

\member{geometry}{TRANSMISSION or REFLECTION.}

\newpage
\subsubsection{Kinetics}
\label{table:kinetics}

Class describing the kinetics camera mode properties. This applies only to kinetics mode cameras.

\begin{table}[h!]\sffamily \footnotesize
\caption{kinetics class members}

\rowcolors{1}{white}{tableBlue}
\begin{tabular}{p{3.5cm} p{2.5cm}  p{2.5cm} }
\toprule
\bfseries Member     & \bfseries Type & \bfseries Example \\
\midrule
name & string & "APS" \\ 
window\_size & integer & 256  \\     
top & integer &  1024  \\
first\_usable\_window & integer & 1  \\
last\_usable\_window & integer & 4  \\
\bottomrule
\end{tabular}
\end{table}

\member{window\_size}{Number of rows in each kinetics window.}

\member{top}{Top pixel.}

\member{first\_usable\_window}{The first usable kinetics window.}

\member{last\_usable\_window}{The last usable kinetics window.}

\newpage
\subsubsection{Setup}
\label{table:xpcssetup}

This class stores XPCS setup parameters.

\begin{table}[h!]\sffamily \footnotesize
\caption{XPCS Setup class members}

\rowcolors{1}{white}{tableBlue}
\begin{tabular}{p{4.5cm} p{3.5cm}  p{3.5cm} }

\toprule
\bfseries Member     & \bfseries Type & \bfseries Example \\
\midrule
beam\_center\_x & float & 0.000 \\   
beam\_center\_y & float & 180.000 \\     
beam\_size\_h & float & 10.000 \\    
beam\_size\_v & float & 18.000 \\    
stage\_zero\_x & float & 199.0 \\    
stage\_zero\_z & float & 185.0 \\    
stage\_x & float & 199.0 \\  
stage\_z & float & 185.0 \\ 
xspec & float & 0.0 \\   
yspec & float & 0.0 \\  
ccdxspec & float & 0.0 \\    
ccdyspec & float & 0.0 \\   
x & float & 50 \\
y & float & 50 \\
angle & float & 50.0 \\
\bottomrule
\end{tabular}
\end{table}

\member{beam\_center\_x}{Pixel location of beam center.}

\member{beam\_center\_y}{Pixel location of beam center.}

\member{beam\_center\_h}{Horizontal beam size on detector.}

\member{beam\_center\_v}{Vertical beam size on detector.}

\member{stage\_zero\_x}{Initial stage location.}

\member{stage\_zero\_y}{Initial stage location.}

\member{stage\_x}{Stage location.}

\member{stage\_y}{Stage location.}

\member{xspec}{...}

\member{yspec}{...}

\member{ccdxspec}{...}

\member{ccdyspec}{...}

\member{x}{...}

\member{y}{...}

\member{angle}{...}

%****************************************************************************


\subsection{APS Sector 8 Process descriptions}
For the APS Sector 8 x-ray photon correlation system we define the following process descriptions:

\newpage
\subsubsection{XPCS}
\label{table:xpcs}

The XPCS class contains all information and parameters required to run a  using the APS cluster.

\begin{table}[h!]\sffamily \footnotesize
\caption{XPCS class members.}

\rowcolors{1}{white}{tableBlue}
\begin{tabular}{p{3.5cm} p{4.0cm}  p{4.5cm} }

\toprule
\bfseries Member     & \bfseries Type & \bfseries Example \\
\midrule
input\_file\_local &  string & "InputFile.imm" \\
output\_file\_local &  string & "OutputFile.imm" \\
input\_file\_remote &  string & "InputFile.imm" \\
output\_file\_remote &  string & "OutputFile.imm" \\
specfile & string & "Specfile.spec" \\
specscan\_dark\_number & integer & \\
specscan\_data\_number & integer & \\
compression & string & "SPARSE" \\
file\_mode & string & "MULTI" \\
delays\_per\_level & integer & \\
lld & float & \\
sigma & float & \\
analysis\_type & string & "DYNAMIC" \\
batches & integer & "2" \\
data\_begin & integer & "1" \\
data\_end & integer & "99998" \\
dark\_begin & integer & "99999" \\
dark\_end & integer & "100000" \\
data\_begin\_todo & integer & "1" \\
data\_end\_todo & integer & "99998" \\
dark\_begin\_todo & integer & "99999" \\
dark\_end\_todo & integer & "100000" \\
mask & 2D array of integers & variable (see Tab. \ref{table:attrib} for attrib.)  \\
dqmap & 2D array of floats & variable (see Tab. \ref{table:attrib} for attrib.)  \\
sqmap & 2D array of floats & variable (see Tab. \ref{table:attrib} for attrib.)  \\
dphimap & 2D array of floats & variable (see Tab. \ref{table:attrib} for attrib.)  \\
sphimap & 2D array of floats & variable (see Tab. \ref{table:attrib} for attrib.)  \\
dqspan & 1D array of floats & variable (see Tab. \ref{table:attrib} for attrib.)  \\
dphispan & 1D array of floats & variable (see Tab. \ref{table:attrib} for attrib.)  \\
sqspan & 1D array of floats & variable (see Tab. \ref{table:attrib} for attrib.)  \\
sphispan & 1D array of floats & variable (see Tab. \ref{table:attrib} for attrib.)  \\
sqlist & 1D array of floats & variable (see Tab. \ref{table:attrib} for attrib.)  \\
dqlist & 1D array of floats & variable (see Tab. \ref{table:attrib} for attrib.)  \\
sphilist & 1D array of floats & variable (see Tab. \ref{table:attrib} for attrib.)  \\
dphilist & 1D array of floats & variable (see Tab. \ref{table:attrib} for attrib.)  \\
normalization\_method & string & "INCIDENT" \\
blemish\_enabled & string & "TRUE" \\
flatfield\_enabled & string & "TRUE" \\
\bottomrule
\end{tabular}
\end{table}

\member{input\_file\_locatl}{Path to the input data file.}

\member{output\_file\_local}{Full path for output data file.}

\member{input\_file\_local}{Path to the input data file.}

\member{output\_file\_local}{Full path for output data file.}

\member{specfile}{Full path to spec file.}

\member{specscan\_dark\_number}{...}

\member{specscan\_data\_number}{...}

\member{compression}{Compression type: either SPARSE or NONSPARSE.}

\member{file\_mode}{MULTI or SINGLE.}

\member{delays\_per\_level}{Delays per level.}

\member{lld}{LLD.}

\member{sigma}{Sigma.}

\member{analysis\_type}{STATIC or DYNAMIC}

\member{batches}{Number of batches.}

\member{data\_begin}{Index of first data frame in .imm file.}

\member{data\_end}{Index of last data frame in .imm file.}

\member{dark\_begin}{Index of first dark data frame in .imm file.}

\member{dark\_end}{Index of last dark data frame in .imm file.}

\member{data\_begin\_todo}{Index of first data frame in .imm file to analyze.}

\member{data\_end\_todo}{Index of last data frame in .imm file to analyze.}

\member{dark\_begin\_todo}{Index of first dark data frame in .imm file to analyze.}

\member{dark\_end\_todo}{Index of last dark data frame in .imm file to analyze.}

\member{mask}{Mask used to exclude image regions during analysis. It's a 2D array of the same dimensions as the collected data, with 0 labeling excluded pixels and 1 labeling included pixels.}

\member{dqmap}{dqmap. 2D array of the same dimensions as the collected data.}

\member{sqmap}{sqmap. 2D array of the same dimensions as the collected data.}

\member{dphimap}{dphimap. 2D array of the same dimensions as the collected data.}

\member{sphimap}{sphimap. 2D array of the same dimensions as the collected data.}

\member{dqspan}{dqspan. 1D array.}

\member{dphispan}{dphispan. 1D array.}

\member{sqspan}{sqspan. 1D array.}

\member{sphispan}{sphispan. 1D array.}

\member{sqlist}{sqlist. 1D array.}

\member{dqlist}{dqlist. 1D array.}

\member{sphilist}{sphilist. 1D array.}

\member{dphilist}{dphilist. 1D array.}

\member{normalization\_method}{Normalization method: INCIDENT or TRANSMITTED or BOTH or NONE.}

\member{blemish\_enabled}{Use detector blemish data.}

\member{flatfield\_enabled}{Use detector flat field data.}
